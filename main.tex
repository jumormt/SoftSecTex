\documentclass[UTF8,a4paper,10pt, twocolumn]{ctexart}
\usepackage[left=2.50cm, right=2.50cm, top=2.50cm, bottom=2.50cm]{geometry}

% -- text font --
% compile using Xelatex

%\setmainfont{Microsoft YaHei}  % 微软雅黑
%\setmainfont{YouYuan}  % 幼圆
%\setmainfont{NSimSun}  % 新宋体
%\setmainfont{KaiTi}    % 楷体
%\setmainfont{SimSun}   % 宋体
%\setmainfont{SimHei}   % 黑体

\usepackage{times}
%\usepackage{mathpazo}
%\usepackage{fourier}
%\usepackage{charter}
%\usepackage{helvet}

\usepackage{amsmath, amsfonts, amssymb} % math equations, symbols
\usepackage[english]{babel}
\usepackage{color}      % color content
\usepackage{graphicx}   % import figures
\usepackage{url}        % hyperlinks
\usepackage{bm}         % bold type for equations
\usepackage{multirow}
\usepackage{booktabs}
\usepackage{epstopdf}
\usepackage{epsfig}
\usepackage{algorithm}
\usepackage{algorithmic}
\renewcommand{\algorithmicrequire}{ \textbf{Input:}}     % use Input in the format of Algorithm
\renewcommand{\algorithmicensure}{ \textbf{Initialize:}} % use Initialize in the format of Algorithm
\renewcommand{\algorithmicreturn}{ \textbf{Output:}}     % use Output in the format of Algorithm

\usepackage{fancyhdr}   % 设置页眉、页脚
%\pagestyle{fancy}
\lhead{}
\chead{}
%\rhead{\includegraphics[width=1.2cm]{fig/ZJU_BLUE.eps}}
\lfoot{}
\cfoot{}
\rfoot{}

\usepackage{draftwatermark}         % 所有页加水印
%\usepackage[firstpage]{draftwatermark} % 只有第一页加水印
%\SetWatermarkText{Water-Mark}           % 设置水印内容
%\SetWatermarkText{\includegraphics{fig/ZJDX-WaterMark.eps}}         % 设置水印logo
%\SetWatermarkLightness{0.9}             % 设置水印透明度 0-1
%\SetWatermarkScale{1}                   % 设置水印大小 0-1

\usepackage{hyperref}   % bookmarks
\hypersetup{colorlinks, bookmarks, unicode} % unicode


\title{仿冒APP识别工具的设计与实现}
\author{ 程潇 2018111027  \thanks{组长}\\
垢宇晴 2018111037  \thanks{组员}\\
王皓 2018110990  \thanks{组员}
}

\date{\today}

\begin{document}
    \maketitle
    \thispagestyle{fancy}

\section{章节 Section} \label{sec:one}
    this is the Section \ref{sec:one}
    输入中文内容 Your text goes here. \\
    ABCDEFGHIJKLMNOPQRSTUVWXYZ \\
    $\mathcal{ABCDEFGHIJKLMNOPQRSTUVWXYZ}$ \\
    abcdefghijklmnopqrstuvwxyz \\
    The quick brown fox jumps over the lazy dog.\\
    1234567890
    \begin{align}
		& ABCDEFGHIJKLMNOPQRSTUVWXYZ \label{eq:alphabet} \\
		& abcdefghijklmnopqrstuvwxyz \\
	& \alpha \beta \gamma \delta \epsilon \varepsilon \zeta \eta \theta \lambda \mu \nu \xi \pi \rho \sigma \tau \upsilon \phi \varphi \chi \psi \omega
	\end{align}
    \begin{align}
	 \begin{bmatrix}
		1 & 2 \\
		3 & 4 \\
	\end{bmatrix}
	 \begin{pmatrix}
	1 & 2 \\
	3 & 4 \\
	\end{pmatrix}
	 \begin{matrix}
	1 & 2 \\
	3 & 4 \\
	\end{matrix}
	\end{align}

    \begin{equation}
	A_{t+1} = \arg\min_A \ \mathcal{L}(A,E_t,\Delta\tau_t,W_t,b_t), \nonumber
	\end{equation}

    \begin{equation}
	\begin{aligned} \label{eq:rasl}
	\min_{A,E,\Delta \tau} \quad & \sum_{i=1}^{N}||A_i||_* + \lambda ||E_i||_1  \\
	\mathrm{s.t.} \quad & D_i \circ \tau_i + \sum_{k=1}^{n_i} J_{ik} \Delta \tau_i \epsilon_k \epsilon_k^T = A_i + E_i, \\
	& i = 1,2,\cdots,N.
	\end{aligned}
	\end{equation}

    \begin{table}[htbp]
		\caption{Title of table} \label{tab:table}
		\centering
		\addtolength{\tabcolsep}{-0mm} % 控制列间距
		\begin{tabular}{ccccc}
			\toprule[0.75pt]	% package booktabs
			\multicolumn{4}{c}{table head} \\
			\midrule[0.5pt]	% package booktabs
			\multirow{4}{*}{text} & 1 & 2 & 3 & 4 \\  % package multirow
			& 5 & 6 & 7 & 8 \\
			\cmidrule[0.5pt]{2-4}	% package booktabs
			& 9 & 10 & 11 & 12 \\
			& 13 & 14 & 15 & 16 \\
			\bottomrule[0.75pt]	% package booktabs
		\end{tabular}
	\end{table}

%    引用: Eq. \eqref{eq:alphabet}, Fig. \ref{figure:zju1},  \\
%    \begin{figure}[htbp]
%		\centering
%		\includegraphics[width=0.2\textwidth]{fig/ZJU_BLACK.eps}
%		\includegraphics[width=0.2\textwidth]{fig/ZJU_BLUE.eps}
%		\caption{figure 1}
%		\label{figure:zju1}
%	\end{figure}

%     Fig. \ref{figure:zju2}
%    \begin{figure}[htbp]
%		\centering
%		\includegraphics[width=0.2\textwidth]{fig/ZJU_BLACK.eps}
%		\includegraphics[width=0.2\textwidth]{fig/ZJU_BLUE.eps}
%		\caption{figure 2}
%		\label{figure:zju2}
%	\end{figure}

    \begin{algorithm}
		\caption{Title of the Algorithm}
		\label{algo:ref}
		\begin{algorithmic}[1]
			\REQUIRE some words.  % this command shows "Input"
			\ENSURE ~\\           % this command shows "Initialized"
			some text goes here ... \\
			\WHILE {\emph{not converged}}
			\STATE ... \\  % line number at left side
			\ENDWHILE
			\RETURN this is the lat part.  % this command shows "Output"
		\end{algorithmic}
	\end{algorithm}

\end{document}